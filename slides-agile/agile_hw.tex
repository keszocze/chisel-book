\input{../slides/common/slides_common}

\newif\ifbook
\input{../shared/chisel}

\title{Agile Hardware Design}
\author{Martin Schoeberl}
\date{\today}
\institute{Technical University of Denmark\\
Embedded Systems Engineering}

\begin{document}

\begin{frame}
\titlepage
\end{frame}


\begin{frame}[fragile]{TODO}
\begin{itemize}
\item Scala/Chisel/Module
\item Functional programming
\item Generator examples
\item Check my normal Chisel book on the topic
\item Show scala-cli
\end{itemize}
\end{frame}

\begin{frame}[fragile]{Overview}
\begin{itemize}
\item Quick overview of Chisel and Scala
\item Installation instructions
\item Links to further reading and material
\end{itemize}
\end{frame}



\begin{frame}[fragile]{Further Reading and Web Resources}
\begin{itemize}
\item \href{https://www.sciencedirect.com/science/article/pii/S014193312500050X}{Scala defined hardware generators for Chisel}
\begin{itemize}
\item Journal article with generator examples
\end{itemize}
\item \href{http://www.imm.dtu.dk/~masca/chisel-book.html}{Chisel book website}
\begin{itemize}
\item Information on Chisel, a bit of Scala
\item Download the free PDF
\end{itemize}
\item \href{http://www2.imm.dtu.dk/courses/02139/}{Digital design course at DTU}
\begin{itemize}
\item Slides on digital design with Chisel
\end{itemize}
\item \href{https://github.com/schoeberl/chisel-lab}{Digital design lab at DTU}
\begin{itemize}
\item Lab material for the digital design course
\item Option to train a bit on Chisel
\end{itemize}
\end{itemize}
\end{frame}



\begin{frame}[fragile]{Chisel}
\begin{itemize}
\item A hardware \emph{construction} language
\begin{itemize}
\item Constructing Hardware in a Scala Embedded Language
\item If it compiles, it is synthesizable hardware 
\item Say goodbye to your unintended latches
\end{itemize}
\item Chisel is not a high-level synthesis language
\item Single source for two targets
\begin{itemize}
\item Cycle accurate simulation (testing)
\item Verilog for synthesis
\end{itemize}
\item Embedded in Scala
\begin{itemize}
\item Full power of Scala available
\item We use Scala to write the generators
\end{itemize}
\item Developed at UC Berkeley
\end{itemize}
\end{frame}

\begin{frame}[fragile]{The C Language Family}

\Tree[.C [
   [.{\bf Verilog} {\bf SystemVerilog} ]
   [.C++  \emph{SystemC}  ]
   [.Java [.Scala {\bf Chisel} ] ]
   [.C\# ] ] ]
 
\end{frame}

\begin{frame}[fragile]{What Language do You Already Know?}
\begin{itemize}
\item ???
\end{itemize}
\end{frame}


\begin{frame}[fragile]{A Small Language}
\begin{itemize}
\item Chisel is a \emph{small} language
\item On purpose
\item Not many constructs to remember
\item The \href{https://github.com/freechipsproject/chisel-cheatsheet/releases/latest/download/chisel_cheatsheet.pdf}{Chisel Cheatsheet} fits on two pages
\item The power comes with Scala for circuit generators
\item With Scala, Chisel can grow with you
\end{itemize}
\end{frame}

\begin{frame}[fragile]{Tool Flow for Chisel Defined Hardware}
\begin{figure}
    \centering
    \includegraphics[scale=0.35]{../figures/flow}
\end{figure}
\end{frame}

\begin{frame}[fragile]{Tooling}
\begin{itemize}
\item ccc
\end{itemize}
\end{frame}


\begin{frame}[fragile]{Chisel is a Hardware Construction Language}
\begin{itemize}
\item The code I showed you looks much like Java code
\item But it is \emph{not} a program in the usual sense
\item It represents a circuit
\item The ``program'' constructs the circuit
\item All statements are ``executed'' in parallel
\item Statement order has \emph{mostly} no meaning
\end{itemize}
\end{frame}

\begin{frame}[fragile]{Free Tools for Chisel and FPGA Design}
\begin{itemize}
\item \href{https://adoptopenjdk.net/}{Java OpenJDK 8} (or later) already installed for Java course
\item \href{https://www.scala-sbt.org/}{sbt, the Scala (and Java) build tool}
\item \href{https://www.jetbrains.com/idea/download/}{IntelliJ (the free Community version)}
\item \href{http://gtkwave.sourceforge.net/}{GTKWave}
\item \href{https://www.xilinx.com/products/design-tools/vivado/vivado-webpack.html}{Vivado WebPACK} already installed from DE1
% \item \href{http://www.altera.com/products/software/quartus-ii/web-edition/qts-we-index.html}{Quartus}
\item Nice to have:
\begin{itemize}
\item make, git
\end{itemize}
\end{itemize}
\end{frame}

\begin{frame}[fragile]{Tool Setup for Different OSs}
\begin{itemize}
\item Windows
\begin{itemize}
\item Use the installers from the websites
\end{itemize}
\item macOS
\begin{itemize}
\item \code{brew install sbt}
\item For the rest, use the installer from the websites
\item Use an Ubuntu VM to run Vivado
\end{itemize}
\item Linux/Ubuntu
\begin{itemize}
\item \code{sudo apt install openjdk-8-jdk git make gtkwave}
\item Install sbt
\item IntelliJ as from the website
\end{itemize}
% \item If setup fails, we have you covered with the databar PCs
\item Instruction details: \url{https://github.com/schoeberl/chisel-lab/blob/master/Setup.md}
\end{itemize}
\end{frame}

\begin{frame}[fragile]{Virtual Machine Setup for Chisel}
\begin{itemize}
\item If setup fails, we have you covered with a Virtual Machine
\item Ubuntu based
\item \href{https://patmos-download.compute.dtu.dk/de2lab.zip}{Ubuntu VM with Vivado} uid: de2lab, pwd: de2lab
\begin{itemize}
\item But this is VERY large (40 GB for the .zip file)
\end{itemize}
%\item \href{http://patmos-download.compute.dtu.dk/patmos-dev.zip}{Ubuntu VM with Quartus} uid: patmos, pwd: patmos
\item Use the  \href{https://www.vmware.com/products/workstation-player.html} {VMWare Workstation Player} (free for Linux and Windows)
\begin{itemize}
\item Use the free VMWare Fusion for macOS
\end{itemize}
\end{itemize}
\end{frame}

\begin{frame}[fragile]{An IDE for Chisel}
\begin{itemize}
\item IntelliJ
\item Install the Scala plugin
\item For IntelliJ: File - New - Project from Existing Sources..., open build.sbt
%\item But you are not compiling with Eclipse\\ and against the Chisel source
\item Show it %(down to the Basys3)
\end{itemize}
\end{frame}


\begin{frame}[fragile]{A Chisel Book}
\begin{figure}
    \centering
    \href{https://github.com/schoeberl/chisel-book}{\includegraphics[scale=0.4]{../cover-small}}
\end{figure}

\begin{itemize}
\item Available in open access (as PDF)
\begin{itemize}
\item Optimized for reading on a tablet (size, hyper links)
\end{itemize}
\item Amazon can do the printout
\end{itemize}
\end{frame}

\begin{frame}[fragile]{Further Information}
\begin{itemize}
\item \url{https://www.chisel-lang.org/}
\item \url{https://github.com/freechipsproject/chisel-cheatsheet/releases/latest/download/chisel_cheatsheet.pdf}
\item \url{https://github.com/ucb-bar/chisel-tutorial}
\item \url{https://github.com/ucb-bar/generator-bootcamp}
%\item Chisel 2 documentation at \url{https://github.com/schoeberl/chisel2-doc}
%\begin{itemize}
%\item Chisel 2.2 Tutorial
%\item Getting Started with Chisel
%\end{itemize}
\item \url{http://groups.google.com/group/chisel-users}
\item \url{https://github.com/schoeberl/chisel-book}
\end{itemize}
\end{frame}


\begin{frame}[fragile]{Your Tasks: Lab Time: Hello World in Chisel}
\begin{itemize}
\item Get a blinking LED working on your FPGA board
\item Clone or download the repository from:
\begin{itemize}
\item \url{https://github.com/schoeberl/chisel-lab}
\end{itemize}
\item Follow the instructions from the lab page
\begin{itemize}
\item Start IntelliJ and follow the instructions from the lab page
\item \code{sbt run}
\item Create a Vivado project
\item Synthesize with the Play button
\item Configure the FPGA with the Programmer button
\end{itemize}
\item {\bf You have your first Chisel design running in an FPGA!}
\begin{itemize}
\item There is also a simulation version available
\end{itemize}
\end{itemize}
\end{frame}



\begin{frame}[fragile]{Change the Design}
\begin{itemize}
\item Use IntelliJ, \code{gedit}, or the editor you like most
\item Source is in \code{.../src/main/scala/Hello.scala}
\item Change blinking frequency
\item Rerun the example
\item Optional:
\begin{itemize}
\item Change to an asymmetric blinking, e.g., 200 ms on every second 
\end{itemize}
\end{itemize}
\end{frame}

\begin{frame}[fragile]{Tiny Tapeout Workshop}
\begin{itemize}
\item What is Tiny Tapeout?
\item Tiny Tapeout can have your design on a real chip
\item For \$ 50 instead of \$ 10000
\item We have a TT workshop at DTU Fr. 15th Feb.
\item Given by Matt Venn
\item \url{https://edu4chip.github.io/ttw2025DTU.html}
\item DTU will buy some PCBs
\item test your chip on a bringup party
\end{itemize}
\end{frame}

\begin{frame}[fragile]{Summary}
\begin{itemize}
\item The world is digital
\item Processors do not get much faster -- we need to design custom hardware
\item We need a modern language for hardware/systems design for efficient/fast development
\item Chisel builds on the power of object-oriented and functional Scala
%\item Chisel is good for hardware generators
% \item You can get started with a hardware design in a special course or 4th semester project
\end{itemize}
\end{frame}

\begin{frame}[fragile]{Let's have a Chat}
\begin{itemize}
\item During your first lab session
\end{itemize}
\end{frame}

\end{document}

\begin{frame}[fragile]{Title}
\begin{itemize}
\item abc
\end{itemize}
\end{frame}
