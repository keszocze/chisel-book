\input{../slides/common/slides_common}

\newif\ifbook
\input{../shared/chisel}

\title{Agile Hardware Design}
\author{Martin Schoeberl}
\date{\today}
\institute{Technical University of Denmark\\
Embedded Systems Engineering}

\begin{document}

\begin{frame}
\titlepage
\end{frame}


\begin{frame}[fragile]{TODO}
\begin{itemize}
\item Scala/Chisel/Module
\item Functional programming
\item Generator examples
\item Check my normal Chisel book on the topic
\item Show scala-cli
\end{itemize}
\end{frame}

\begin{frame}[fragile]{Lecture 1}
\begin{itemize}
\item Quick overview of Chisel and Scala
\item Functional programming in Scala
\item Links to further reading and material
\end{itemize}
\end{frame}


\begin{frame}[fragile]{Chisel}
\begin{itemize}
\item A hardware \emph{construction} language
\begin{itemize}
\item Constructing Hardware in a Scala Embedded Language
\item If it compiles, it is synthesizable hardware 
\item Say goodbye to your unintended latches
\end{itemize}
\item Chisel is not a high-level synthesis language
\item Single source for two targets
\begin{itemize}
\item Cycle accurate simulation (testing)
\item Verilog for synthesis
\end{itemize}
\item Embedded in Scala
\begin{itemize}
\item Full power of Scala available
\item We use Scala to write the generators
\end{itemize}
\item Developed at UC Berkeley
\end{itemize}
\end{frame}

\begin{frame}[fragile]{The C Language Family}

\Tree[.C [
   [.{\bf Verilog} {\bf SystemVerilog} ]
   [.C++  \emph{SystemC}  ]
   [.Java [.Scala {\bf Chisel} ] ]
   [.C\# ] ] ]
 
\end{frame}

%\begin{frame}[fragile]{What Language do You Already Know?}
%\begin{itemize}
%\item ???
%\end{itemize}
%\end{frame}


\begin{frame}[fragile]{A Small Language}
\begin{itemize}
\item Chisel is a \emph{small} language
\item On purpose
\item Not many constructs to remember
\item The \href{https://github.com/freechipsproject/chisel-cheatsheet/releases/latest/download/chisel_cheatsheet.pdf}{Chisel Cheatsheet} fits on two pages
\item The power comes with Scala for circuit generators
\item With Scala, Chisel can grow with you
\end{itemize}
\end{frame}

\begin{frame}[fragile]{Tool Flow for Chisel Defined Hardware}
\begin{figure}
    \centering
    \includegraphics[scale=0.35]{../figures/flow}
\end{figure}
\end{frame}

\begin{frame}[fragile]{Chisel is a Hardware Construction Language}
\begin{itemize}
\item The code I showed you looks much like Java code
\item But it is \emph{not} a program in the usual sense
\item It represents a circuit
\item The ``program'' constructs the circuit
\item All statements are ``executed'' in parallel
\item Statement order has \emph{mostly} no meaning
\end{itemize}
\end{frame}

\begin{frame}[fragile]{A Chisel Book}
\begin{figure}
    \centering
    \href{https://github.com/schoeberl/chisel-book}{\includegraphics[scale=0.4]{../cover-small}}
\end{figure}

\begin{itemize}
\item Available in open access (as PDF)
\begin{itemize}
\item Optimized for reading on a tablet (size, hyper links)
\end{itemize}
\item Amazon can do the printout
\end{itemize}
\end{frame}

\begin{frame}[fragile]{Further Reading and Web Resources}
\begin{itemize}
\item \href{https://www.sciencedirect.com/science/article/pii/S014193312500050X}{Scala defined hardware generators for Chisel}
\begin{itemize}
\item Journal article with generator examples
\end{itemize}
\item \href{http://www.imm.dtu.dk/~masca/chisel-book.html}{Chisel book website}
\begin{itemize}
\item Information on Chisel, a bit of Scala
\item Download the free PDF
\end{itemize}
\item \href{http://www2.imm.dtu.dk/courses/02139/}{Digital design course at DTU}
\begin{itemize}
\item Slides on digital design with Chisel
\end{itemize}
\item \href{https://github.com/schoeberl/chisel-lab}{Digital design lab at DTU}
\begin{itemize}
\item Lab material for the digital design course
\item Option to train a bit on Chisel
\end{itemize}
\end{itemize}
\end{frame}


\begin{frame}[fragile]{Lab 1}
\begin{itemize}
\item I assume that you installed all tools and did lab0 as homework
\item Functional programming in Scala
\item Links to further reading and material
\end{itemize}
\end{frame}

\begin{frame}[fragile]{Lecture 2}
\begin{itemize}
\item More on Chisel
\item Functional programming for generators
\end{itemize}
\end{frame}

\begin{frame}[fragile]{Lab 2}
\begin{itemize}
\item Write an arbitration circuit (with \code{treeReduce()})
\item Introduce some fairness to the arbitration circuit
\end{itemize}
\end{frame}

\begin{frame}[fragile]{Lecture 3}
\begin{itemize}
\item Hardware generators
\end{itemize}
\end{frame}

\begin{frame}[fragile]{Lab 3}
\begin{itemize}
\item Maybe the search circuit
\end{itemize}
\end{frame}

\begin{frame}[fragile]{Summary}
\begin{itemize}
\item The world is digital
\item Processors do not get much faster -- we need to design custom hardware
\item We need a modern language for hardware/systems design for efficient/fast development
\item Chisel builds on the power of object-oriented and functional Scala
%\item Chisel is good for hardware generators
% \item You can get started with a hardware design in a special course or 4th semester project
\end{itemize}
\end{frame}


\end{document}

\begin{frame}[fragile]{Title}
\begin{itemize}
\item abc
\end{itemize}
\end{frame}
