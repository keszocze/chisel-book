\input{common/slides_common}

\newif\ifbook
\input{../shared/chisel}

\title{Finite-State Machines}
\author{Martin Schoeberl}
\date{\today}
\institute{Technical University of Denmark\\
Embedded Systems Engineering}

\begin{document}

\begin{frame}
\titlepage
\end{frame}


\begin{frame}[fragile]{Overview}
\begin{itemize}
\item A bit of testing (repetition)
\item Fun with counters
\item Finite-state machines
\item Collection with \code{Vec}
\end{itemize}
\end{frame}

\begin{frame}[fragile]{Organization and Lab Work}
\begin{itemize}
\item How did the testing lab work? Did you find both bugs?
\item This week is the 7-segment decoder
\item This is part of the lab grade -- show it a TA
\begin{itemize}
\item Deadline is next week (100 \%)
\item In two weeks (27/3) only 50 \%
\end{itemize}
\item Register your group to get a number!
\begin{itemize}
\item Group name: Group work
\end{itemize}
\end{itemize}
\end{frame}

\begin{frame}[fragile]{Testing with Chisel}
\begin{itemize}
\item A test contains
\begin{itemize}
\item a device under test (DUT) and
\item the testing logic
\end{itemize}
\item Set input values with \code{poke}
\item Advance the simulation with \code{step}
\item Read the output values with \code{peek}
\item Compare the values with \code{expect}
\item Import following packages
\shortlist{../code/test_import.txt}
\end{itemize}
\end{frame}

\begin{frame}[fragile]{An Example DUT}
\begin{itemize}
\item A device-under test (DUT)
\item Just 2-bit AND logic
\shortlist{../code/test_dut.txt}
\end{itemize}
\end{frame}

\begin{frame}[fragile]{A ChiselTest}
\begin{itemize}
\item Extends class \code{AnyFlatSpec} with \code{ChiselScalatestTester}
\item Has the device-under test (DUT) as parameter of the \code{test()} function
\item Test function contains the test code
\item Testing code can use all features of Scala
\item Is placed in \code{src/test/scala}
\item Is run with \code{sbt test}
\end{itemize}
\end{frame}


\begin{frame}[fragile]{Testing with \code{expect()}}
\begin{itemize}
\item Poke values and \code{expect} some output
\shortlist{../code/test_bench.txt}
\end{itemize}
\end{frame}

\begin{frame}[fragile]{Call the Tester for Waveform Generation}
\begin{itemize}
\item The complete test
\item Note the \code{.withAnnotations(Seq(WriteVcdAnnotation)}
\end{itemize}
\begin{chisel}
class Count6WaveSpec extends AnyFlatSpec with ChiselScalatestTester {
  "CountWave6 " should "pass" in {
    test(new Count6).withAnnotations(Seq(WriteVcdAnnotation)) { dut =>
      dut.clock.step(20)
    }
  }
}
\end{chisel}
\end{frame}

\begin{frame}[fragile]{Display Waveform with GTKWave}
\begin{itemize}
\item Run the tester: \code{sbt test}
\item Locate the .vcd file in test\_run\_dir/...
\item Start GTKWave
\item Open the .vcd file with
\begin{itemize}
\item File -- Open New Tab
\end{itemize}
\item Select the circuit
\item Drag and drop the interesting signals
\end{itemize}
\end{frame}





\begin{frame}[fragile]{Counters as Building Blocks}
\begin{itemize}
\item Counters are versatile tools
\item Count events
\item Generate timing ticks
\item Generate a one-shot timer
\end{itemize}
\end{frame}

\begin{frame}[fragile]{Counting Up and Down}
\begin{itemize}
\item Up:
\shortlist{../code/when_counter.txt}
\item Down:
\shortlist{../code/down_counter.txt}
\end{itemize}
\end{frame}

\begin{frame}[fragile]{Generating Timing with Counters}
\begin{itemize}
\item Generate a \code{tick} at a lower frequency
\item We used it in Lab 1 for the blinking LED
\item Use it today for the lab exercise
\item Use it for driving the free running counter at 2 Hz
\end{itemize}
\begin{figure}
  \includegraphics[scale=0.8]{../figures/tick_wave}
\end{figure}
\end{frame}


\begin{frame}[fragile]{The Tick Generation}
\shortlist{../code/sequ_tick_gen.txt}
\end{frame}

\begin{frame}[fragile]{Using the Tick}
\begin{itemize}
\item A counter running at a \emph{slower frequency}
\item By using the \code{tick} as an enable signal
\end{itemize}
\shortlist{../code/sequ_tick_counter.txt}
\end{frame}

\begin{frame}[fragile]{The \emph{Slow} Counter}
\begin{itemize}
\item Incremented every \code{tick}
\end{itemize}
\begin{figure}
  \includegraphics[scale=0.8]{../figures/tick_count_wave}
\end{figure}
\end{frame}

\begin{frame}[fragile]{A Timer}
\begin{itemize}
\item Like a kitchen timer
\item Start by loading a timeout value
\item Count down till 0
\item Assert \code{done} when finished
\end{itemize}
\end{frame}

\begin{frame}[fragile]{One-Shot Timer}
\begin{figure}
  \includegraphics[scale=\scale]{../figures/timer}
\end{figure}
\end{frame}

\begin{frame}[fragile]{One-Shot Timer}
\shortlist{../code/timer.txt}
\end{frame}

\begin{frame}[fragile]{A 4 Stage Shift Register}
\begin{figure}
  \includegraphics[scale=\scale]{../figures/shiftregister}
\end{figure}
\shortlist{../code/shift_register.txt}
\end{frame}


\begin{frame}[fragile]{A Shift Register with Parallel Output}
\begin{figure}
  \includegraphics[scale=\scale]{../figures/shiftreg-paraout}
\end{figure}
\shortlist{../code/shift_paraout.txt}
\end{frame}

\begin{frame}[fragile]{A Shift Register with Parallel Load}
\begin{figure}
  \includegraphics[scale=0.5]{../figures/shiftreg-paraload}
\end{figure}
\shortlist{../code/shift_paraload.txt}
\end{frame}



\begin{frame}[fragile]{A Simple Circuit}
\begin{itemize}
\item What does the following circuit?
\item Is this related to finite-state machines?
\end{itemize}
\begin{figure}
  \includegraphics[scale=\scale]{../figures/fsm-rising}
\end{figure}
\end{frame}

\begin{frame}[fragile]{Before the Break}
\begin{itemize}
\item Let us talk about AI tools
\end{itemize}
\end{frame}

\begin{frame}[fragile]{ChatGPT/Copilot}
\begin{itemize}
\item Maybe useful to learn a language
\item Sometimes ChatGPT uses old Chisel constructs
\item Sometimes it is even plain wrong
\item DTU does not allow the usage at the exam
\begin{itemize}
\item We will have the exam online without Internet access
\item But you could run DeepSeek locally
\end{itemize}
\item My personal opinion:
\begin{itemize}
\item It is just a new tool
\item We cannot really (and shall not) disallow tools (grammar check, calculator, programming,...)
\item We will need to learn how to deal with it
\item I use Copilot in my editor
\item Sometimes I use ChatGPT to rewrite text
\end{itemize}
\end{itemize}
\end{frame}

\begin{frame}[fragile]{ChatGPT/Copilot}
\begin{itemize}
\item I tried to solve todays lab with ChatGPT (3.5)
\item It took me 4 prompts and one correction prompt
\item The result looks correct
\item Done in 10 minutes
%\begin{itemize}
%\item Show it
%\end{itemize}
\item Can one learn to code with ChatGPT? Even better?
\item I do not know
\end{itemize}
\end{frame}

\begin{frame}[fragile]{Break}
\begin{itemize}
\item 10 minutes
\end{itemize}
\end{frame}


\begin{frame}[fragile]{Finite-State Machine (FSM)}
\begin{itemize}
\item Has a register that contains the state
\item Has a function to computer the next state
\begin{itemize}
\item Depending on current state and input
\end{itemize}
\item Has an output depending on the state
\begin{itemize}
\item And maybe on the input as well
\end{itemize}
\item Every synchronous circuit can be considered a finite state machine
\item However, sometimes the state space is a little bit too large
\end{itemize}
\end{frame}

\begin{frame}[fragile]{Basic Finite-State Machine}
\begin{itemize}
\item A state register
\item Two combinational blocks
\end{itemize}
\begin{figure}
  \includegraphics[scale=\scale]{../figures/fsm}
\end{figure}
\end{frame}

\begin{frame}[fragile]{State Diagram}
\begin{figure}
  \includegraphics[scale=\scale]{../figures/state-diag-moore}
\end{figure}
\begin{itemize}
\item States and transitions depending on input values
\item Example is a simple alarm FSM
\item Nice visualization
\item Will not work for large FSMs
\item What is the transition in \code{orange} when there is \code{clear} and \code{bad event}?
\item Complete code in the Chisel book
\end{itemize}
\end{frame}


\begin{frame}[fragile]{State Table for the Alarm FSM}
\begin{table}
\centering
\begin{tabular}{ccccc}
\toprule
& \multicolumn{2}{c}{Input} \\
\cmidrule{2-3}
State &  Bad event & Clear & Next state & Ring bell \\
\midrule
green & 0 & 0 & green & 0 \\
green & 1 & - & orange & 0 \\
orange & 0 & 0 & orange & 0 \\
orange & 1 & - & red & 0 \\
orange & 0 & 1 & green & 0 \\
red & 0 & 0 & red & 1 \\
red & 0 & 1 & green & 1 \\
\bottomrule
\end{tabular}
\label{tab:state:table}
\end{table}
\end{frame}

\begin{frame}[fragile]{The Input and Output of the Alarm FSM}
\begin{itemize}
\item Two inputs and one output
\end{itemize}
\shortlist{../code/simple_fsm_io.txt}
\end{frame}

\begin{frame}[fragile]{Encoding the State}
\begin{itemize}
\item We can optimize state encoding
\item Two common encodings are: binary and one-hot
\item We leave it to the synthesize tool
\item Use symbolic names with \code{ChiselEnum}
\end{itemize}
\shortlist{../code/simple_fsm_states.txt}
\end{frame}

\begin{frame}[fragile]{Start the FSM}
\begin{itemize}
\item We have a starting state on reset
\end{itemize}
\shortlist{../code/simple_fsm_register.txt}
\end{frame}


\begin{frame}[fragile]{The Next State Logic}
\shortlist{../code/simple_fsm_next.txt}
\end{frame}

\begin{frame}[fragile]{The Output Logic}
\shortlist{../code/simple_fsm_output.txt}
\end{frame}

\begin{frame}[fragile]{Summary of the Alarm Example}
\begin{itemize}
\item Three elements:
\begin{enumerate}
\item State register
\item Next state logic
\item Output logic
\end{enumerate}
\item This was a so-called Moore FSM
\item There is also a FSM type called Mealy machine
\end{itemize}
\end{frame}

\begin{frame}[fragile]{A so-called Mealy FSM}
\begin{itemize}
\item Similar to the former FSM
\item Output also depends in the input
\item It can react \emph{faster}
\item Less composable (draw it)
\end{itemize}
\begin{figure}
  \includegraphics[scale=\scale]{../figures/mealy}
\end{figure}
\end{frame}

\begin{frame}[fragile]{The Mealy FSM for the Rising Edge}
\begin{itemize}
\item That was our starting example
\item Output is also part of the transition arrows
\end{itemize}
\begin{figure}
  \includegraphics[scale=\scale]{../figures/state-diag-mealy}
\end{figure}
\end{frame}

\begin{frame}[fragile]{The Mealy Solution}
\begin{itemize}
\item Show code in IntelliJ as it is too long for slides
\end{itemize}
\end{frame}

\begin{frame}[fragile]{State Diagram for the Moore Rising Edge Detection}
\begin{itemize}
\item We need three states
\item Show code in IntelliJ
\end{itemize}
\begin{figure}
  \includegraphics[scale=\scale]{../figures/state-diag-rising-moore}
\end{figure}
\end{frame}

\begin{frame}[fragile]{Comparing with a Timing Diagram}
\begin{itemize}
\item Moore is delayed one clock cycle compared to Mealy
\end{itemize}
\begin{figure}
  \includegraphics[scale=1]{../figures/rising}
\end{figure}
\end{frame}

\begin{frame}[fragile]{What is Better?}
\begin{itemize}
\item It depends ;-)
\item Moore is on the save side
\item Moore is composable
\item Mealy has \emph{faster} reaction
\item Both are tools in you toolbox
\item Keep it simple with your vending machine and use a Moore FSM
\end{itemize}
\end{frame}

\begin{frame}[fragile]{Another Simple FSM}
\begin{itemize}
\item a FSM for a single word buffer
\item Just two symbols for the state machine
\end{itemize}
\begin{chisel}
  object State extends ChiselEnum {
    val empty, full = Value
  }
\end{chisel}
\end{frame}

\begin{frame}[fragile]{Finite State Machine for a Buffer}
\begin{chisel}
  object State extends ChiselEnum {
    val empty, full = Value
  }
  import State._

  val stateReg = RegInit(empty)
  val dataReg = RegInit(0.U(8.W))

  when(stateReg === empty) {
    when(io.in.valid) {
      dataReg := io.in.bits
      stateReg := full
    }
  } .otherwise { // full
    when(io.out.ready) {
      stateReg := empty
    }
  }
\end{chisel}
\begin{itemize}
\item A simple buffer for a bubble FIFO
\end{itemize}
\end{frame}


\begin{frame}[fragile]{Group Signals with a \code{Bundle}}
\begin{itemize}
\item Group signals that belong to each other
\item Reference as a whole
\item Individual fields accessed by their name
\item E.g., \code{ref.field}
\end{itemize}
\shortlist{../code/bundle.txt}
\end{frame}

\begin{frame}[fragile]{A Collection of Signals with \code{Vec}}
\begin{itemize}
\item Chisel \code{Vec} is a collection of signals of the same type
\item The collection can be accessed by an index
\item Similar to an array in other languages
\item Wrap into a \code{Wire()} for combinational logic
\item Wrap into a \code{Reg()} for a collection of registers
\end{itemize}
\shortlist{../code/vec.txt}
\end{frame}

\begin{frame}[fragile]{Using a \code{Vec}}
\shortlist{../code/vec_access.txt}
\begin{itemize}
\item Reading from an \code{Vec} is a multplexer
\item We can put a \code{Vec} into a \code{Reg}
\end{itemize}
\shortlist{../code/reg_file.txt}
\noindent An element of that register file is accessed with an index and used as a normal register.

\shortlist{../code/reg_file_access.txt}
\end{frame}


\begin{frame}[fragile]{Mixing Vecs and Bundles}
\begin{itemize}
\item We can freely mix bundles and vectors
\item When creating a vector with a bundle
type, we need to pass a prototype for the vector fields. Using our
\code{Channel}, which we defined above, we can create a vector of channels with:
\end{itemize}
\shortlist{../code/vec_bundle.txt}
\begin{itemize}
\item A bundle may as well contain a vector
\end{itemize}
\shortlist{../code/bundle_vec.txt}
\end{frame}

\begin{frame}[fragile]{Today's Lab}
\begin{itemize}
\item This is the start of the graded group work
\begin{itemize}
\item Part of your grade
\item Please register your group in DTU Learn
\end{itemize}
\item Binary to 7-segment decoder
\item First part of your vending machine
\item Just a single digit, only combinational logic (start)
\item Use the nice tester provided to develop the circuit
\item Then synthesize it for the FPGA
\item Test with switches
\item Then add two counters to display the counting at 2 Hz
\item Show a TA your working design
\item \href{https://github.com/schoeberl/chisel-lab/tree/master/lab5}{Lab 5}
\end{itemize}
\end{frame}



\begin{frame}[fragile]{Summary}
\begin{itemize}
\item Waveform testing is the way to develop/debug
\item Counters are important tools, e.g., to generate timing
\item Finite-state machines are another tool of the trade
\item Two types: Moore and Mealy
\item A Chisel \code{Vec} is the hardware version of an array
\end{itemize}
\end{frame}


\end{document}

%\begin{frame}[fragile]{xxx}
%\begin{itemize}
%\item yyy
%\end{itemize}
%\end{frame}
